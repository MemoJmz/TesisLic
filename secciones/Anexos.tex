\chapter{Anexos}\label{cap:anexos}

\section{Esquema General de las Leyes de Conservación}
Sea $\Omega \subset \mathbb{R}^d$ arbitrario, abierto y acotado; con frontera $\partial \Omega$ suave -podemos aplicar el teorema de la divergencia.

La \textit{ley de balance} establece que:
\textit{Cantidades extensivas (masa, momento, energía, velcidad, etc.) tiene un balance entre producción dentro de $\Omega$ con la variación de las cantidades extensivas en el tiempo y el flujo de elas a través de $\partial \Omega$}.
Denotamos por
\[
u(x,t) = \begin{pmatrix}
			u_1(x,t) \\
			\vdots	 \\
			u_n(x,t) 
		 \end{pmatrix} \qquad \qquad x \in \Omega, \quad t \geq 0
\]
con $u(x,t) \in \R^n$ al vector de densidades por unidad de volumen de las variables de estado.

Suponemos que $u(x,t) \in U \subset \R^n$ es abierto y conexo -$U$ es llamado \textit{dominio de variables de estado}. 

Definimos
\[
M(t) := \int_{\Omega} u(x,t) dx
\]
como la masa total de las variables conservadas en $\Omega$ a tiempo $t>0$. Suponemos que existe un tensor $F \in \R^{n\times d}$ tal que la cantidad de $u$ ue fluye a través de un elemento $dS \subset \partial \Omega$, cuya normal exterior es $\hat{n} \in \R^d$, -$|\hat{n}| = 1$- esta dada por
\[
- F \hat{n} d S_x \in \R^n
\] 
El flujo de cada variable de estado $u_j$ es $-f^j \cdot \hat{n} dS_x$ donde $F = (f^1 \cdots f^d) \in \R^{n\times d}$, esto es, $f^j \in \R^{d \times 1}$ es la j-ésima columna de $F$.

Luego, la masa total de $u$ que fluye a través de la frontera es
\[
-\int_{\partial \Omega} F \hat{n} dS_x
\]
Ahora, la \textit{producción} es la interacción del sistema con campos externos o términos de producción, representados por la función $(x,t) \mapsto g(x,t) \in \R^n$. La componente $g_j$ es la densidad por unidad de volumen del término de balance para la cantidad $u_j$.

Así, en términos matemáticos, formulamos el \textit{Principio de Balance}
\begin{align*}
\frac{d}{dt} M(t) &= \frac{d}{dt} \int_{\Omega} u(x,t) dx \\
				 &= - \int_{\partial \Omega} F \hat{n} dS_x + \int_{\Omega} d dx
\end{align*}
Para cada componente, $k$ fijo, $1\leq k \leq n$,
\begin{align*}
\int_{\Omega} g_k dx &= \frac{d}{dt} \int_{\Omega} u_k(x,t) dx + \int_{\partial\Omega} \sum_{j=1}^d f_k^j n_j dS_x \\
\text{Por el teorema de la divergencia} \\
			&= \frac{d}{dt} \int_{\Omega} u_k(x,t) + \int_{\Omega} \text{div}_x \begin{pmatrix}
								f_k^1 \\
								\vdots \\
								f_k^d
							\end{pmatrix} dx \\
			&= \int_{\Omega} \left[ \partial_t u_k + \sum_{j=1}^d \partial_{x_j} (f_k^j) \right] dx
\end{align*}
Entonces, para todo volumen $\Omega$ arbitrario
\begin{align*}
\int_{\Omega} \left[ \partial_t u_k + \sum_{j=1}^d \partial_{x_j} (f_k^j) - g_k \right] dx = 0	\\
\iff \int_{\Omega} \left[ \partial_t u + \sum_{j=1}^d \partial_{x_j} (f^j) - g \right] dx = 0,
\end{align*}
en $\R^n$.

Suponiendo que el integrando es continuo concluimos que
\[
\partial_t u + \sum_{j=1}^d \partial_{x_j} f^j = g
\]
En la modelación, sistemas de leyes de conservación significa que $f^j = f^j(u)$. Si
\begin{tabular}{r l}
$g = g(u)$ & ley de balance \\
$g = g(x,t)$ & forzamiento externo \\
$g \equiv 0$ & ley de conservación.
\end{tabular}



\section{Ecuación de Burgers}
Una ecuación cuasilieneal de primer orden muy importante es la \textit{ecuación de Burgers}
\[ u_t + u \thinspace u_x = 0, \]
la cual es una EDP no lineal que se obtiene al sustituir en la ley de de conservación evolutiva $f \equiv 0$ y $\phi = u^2 /2$.

Más generalmente la ecuación 
\[
u_t + g(u) \thinspace u_x = 0
\]
se ha usado para modelar el tráfico de automóviles en una vía muy transitada, la dinámica de ciertos gases o en el modelado de la esquistosomiasis. En le caso del tráfico $\nabla \phi$ representa la cnatidad de automóviles que pasan por un punto dado y es función de la densidad de automóviles $u$. Diferentes funciones $g$ se han econtrado experimentalmente y ejemplos sencillos han sido extensivamente estudiados, por ejemplo,
\begin{align*}
g(u) &= cu (1-u) \\
g(u) &= cu \qquad \qquad \text{flujo lineal} \\
g(u) &= u^2/2 \qquad \qquad \text{flujo cuadrático}.
\end{align*}
La ecuación
\begin{equation*}
u_t + g(u) \thinspace u_x =0
\end{equation*}
puede ser resuelta por el método de las características.

Tenemos el sistema característico
\begin{align*}
\frac{\,dt}{\,d\tau} (s, \tau) &= 1,\\
\frac{\,dx}{\,d\tau} (s, \tau) &= g(u(s, \tau)),\\
\frac{\,du}{\,d\tau} (s, \tau) &= 0.
\end{align*}
Así $u(s,\tau)=c_1 , t(s,\tau) = \tau+c_2$ y $x (s,\tau)=g(c_1) \tau + c_3$. 
Si se dan las condiciones iniciales para $\tau=0$
\begin{align*}
x(s,0)=x_0(s), \quad t(s,0)=t_0(s) \quad \text{y} \quad u(s,0)=u_0(s),
\end{align*}
se obtiene la solución
\begin{eqnarray*}
t(s,\tau) &=& \tau + t_0(s) \\
x(s,\tau) &=& g(u_0(s)) \tau + x_0(s) \\
u(s,\tau) &=& u_0(s).
\end{eqnarray*}

\textbf{Ejemplo:} Resuelva el problema de valor inicial
\begin{equation*}
u_t + u u_x = 0
\end{equation*}
$$
u(x,0) = \begin{cases}
0,	&	x < 0, \\
1,	&	x \geq 0.
\end{cases}
$$
Para este problema la condición inicial puede ser representada paramétricamente como
\[
t(s,0)=0, \quad x(s,0)=s, \quad u(s,0)=u(x(s,0),0)
\]
Para las características tenemos el sistema
\begin{align*}
\frac{\,dt}{\,d\tau} (s, \tau) &= 1,\\
\frac{\,dx}{\,d\tau} (s, \tau) &= u(s,\tau),\\
\frac{\,du}{\,d\tau} (s, \tau) &= 0.
\end{align*}
Al integrar se obtiene
\begin{eqnarray*}
t(s,\tau) &=& \tau, \\
u(s,\tau) &=& u(s,0) = \begin{cases}
0, \quad s<0, \\
1, \quad s\geq0. \end{cases}\\
x(s,\tau) &=& s+\tau \cdot \begin{cases}
0, \quad s<0, \\
1, \quad s\geq0. \end{cases}
\end{eqnarray*}

Las características pueden construirse ahora partiendo de las condiciones iniciales; por ejemplo, la característica que pasa por el punto $(-2,0)$ es $x=-2$ y la característica que pasa por $(2,0)$ es $x=2+t$. Entonces tenemos $u(-2,t)=0$ y $u(2+t,t)=1$, para $t>0$. De esta manera se ha obtenido la solución del problema en forma paramétrica.














