\chapter{Anexos}\label{cap:anexos}

\section{Esquema General de las Leyes de Conservación}

Consideramos un medio líquido, gas o sólido que ocupa una región o dominio (abierto y conexo) $\Omega \subset \mathbb{R}^n$, donde $n$ es la dimensión del espacio.

Denotamos
\[
u = u(x,t), \qquad \qquad x \in \Omega, \quad t \geq 0
\]
a una función, llamada \textit{función de estado,} la cual dependiendo del problema podrá representar temperatura o bien, la concentración de una sustancia, etc. Para el análisis de las leyes de conservación, se requiere un dominio de balance escogido arbitrariamente $\Omega' \subset \Omega,$ y un intervalo de tiempo arbitrario $[t_1, t_2]$. 

En general, $u$ puede ser función vectorial: $u: \Omega \longrightarrow \mathbb{R}^n$. En particular para $u$ es una función escalar para $n=1$.

El caso $u$ escalar, incluye también, en general, una función fuente escalar $f=f(x,t), f:\Omega \times [0,t) \longrightarrow \mathbb{R}$ y un campo vectorial flujo, $\phi(x,t), \phi:\Omega \times [0,t) \longrightarrow \mathbb{R}^n$.

Por ejemplo, si $u$ representa la temperatura, $f$ podría representar una fuente interna de calor, por ejemplo, una corriente eléctrica en el alambre y $\phi$ representa una ley física que determina la manera como cambia $u$, por ejemplo, la ley de calor de Fourier.

La ley básica de balance establece que el cambio total de la cantidad $u$ contenida en $\Omega'$ entre los tiempos $t_1$ y $t_2$ debe igualar el flujo total a través de la frontera $\Omega'$ entre los tiempos $t_1$ y $t_2$ y el incremento o decremento de la cantidad $u$ produciendo por la fuente $f$, dentro de $\Omega'$ en el mismo intervalo de tiempo. En forma matemática esto queda expresado como
\[
\int_{\Omega'} u(x,t_2) \,dx - \int_{\Omega'} u(x,t_1) \,dx = - \int_{t_1}^{t_2} \int_{\partial \Omega'} \phi (x,t) \cdot \textbf{n} \,dS \,dt + \int_{t_1}^{t_2} \int_{\Omega'} f(x,t) \,dx \,dt.
\]
Si se supone que $u$ tiene primera derivada continua respecto de $t$, por medio del teorema fundamental del cálculo y del teorema de Fubini se obtiene
\[
\int_{\Omega'} \frac{\partial}{\partial t} u(x,t) \,dx = - \int_{\partial \Omega'} \phi(x,t) \cdot \textbf{n} \,dS + \int_{\Omega'} f(x,t) \,dx.
\]
Al utilizar el teorema de la divergencia, podemos escribir
\[
\int_{\Omega'} \right( \frac{\partial}{\partial} u(x,t) + \text{div} \thinspace \phi(x, t)- f(x,t) \left) \,dx = 0.
\]

La cual es \textit{la forma global o integral de la ley de conservación evolutiva}. Si se supone la continuidad del integrando, dado que la región $\Omega' \subset \Omega$ fue arbitraria, se obtiene \textif{la forma local o diferencial de la ley de conservación evolutiva}
\[
u_t (x,t) + \text{div} \thinspace \phi(x,t)-f(x,t) = 0.
\]  

\section{Ecuación de Burgers}
Una ecuación cuasilieneal de primer orden muy importante es la \textit{ecuación de Burgers}
\[ u_t + u \thinspace u_x = 0, \]
la cual es una EDP no lineal que se obtiene al sustituir en la ley de de conservación evolutiva $f \equiv 0$ y $\phi = u^2 /2$.

Más generalmente la ecuación 
\[
u_t + g(u) \thinspace u_x = 0
\]
se ha usado para modelar el tráfico de automóviles en una vía muy transitada, la dinámica de ciertos gases o en el modelado de la esquistosomiasis. En le caso del tráfico $\nabla \phi$ representa la cnatidad de automóviles que pasan por un punto dado y es función de la densidad de automóviles $u$. Diferentes funciones $g$ se han econtrado experimentalmente y ejemplos sencillos han sido extensivamente estudiados, por ejemplo,
\begin{align*}
g(u) &= cu (1-u) \\
g(u) &= cu \qquad \qquad \text{flujo lineal} \\
g(u) &= u^2/2 \qquad \qquad \text{flujo cuadrático}.
\end{align*}
La ecuación
\begin{equation*}
u_t + g(u) \thinspace u_x =0
\end{equation*}
puede ser resuelta por el método de las características.

Tenemos el sistema característico
\begin{align*}
\frac{\,dt}{\,d\tau} (s, \tau) &= 1,\\
\frac{\,dx}{\,d\tau} (s, \tau) &= g(u(s, \tau)),\\
\frac{\,du}{\,d\tau} (s, \tau) &= 0.
\end{align*}
Así $u(s,\tau)=c_1 , t(s,\tau) = \tau+c_2$ y $x (s,\tau)=g(c_1) \tau + c_3$. 
Si se dan las condiciones iniciales para $\tau=0$
\begin{align*}
x(s,0)=x_0(s), \quad t(s,0)=t_0(s) \quad \text{y} \quad u(s,0)=u_0(s),
\end{align*}
se obtiene la solución
\begin{eqnarray*}
t(s,\tau) &=& \tau + t_0(s) \\
x(s,\tau) &=& g(u_0(s)) \tau + x_0(s) \\
u(s,\tau) &=& u_0(s).
\end{eqnarray*}

\textbf{Ejemplo:} Resuelva el problema de valor inicial
\begin{equation*}
u_t + u u_x = 0
\end{equation*}
$$
u(x,0) = \begin{cases}
0,	&	x < 0, \\
1,	&	x \geq 0.
\end{cases}
$$
Para este problema la condición inicial puede ser representada paramétricamente como
\[
t(s,0)=0, \quad x(s,0)=s, \quad u(s,0)=u(x(s,0),0)
\]
Para las características tenemos el sistema
\begin{align*}
\frac{\,dt}{\,d\tau} (s, \tau) &= 1,\\
\frac{\,dx}{\,d\tau} (s, \tau) &= u(s,\tau),\\
\frac{\,du}{\,d\tau} (s, \tau) &= 0.
\end{align*}
Al integrar se obtiene
\begin{eqnarray*}
t(s,\tau) &=& \tau, \\
u(s,\tau) &=& u(s,0) = \begin{cases}
0, \quad s<0, \\
1, \quad s\geq0. \end{cases}\\
x(s,\tau) &=& s+\tau \cdot \begin{cases}
0, \quad s<0, \\
1, \quad s\geq0. \end{cases} 
\end{eqnarray*}

Las características pueden construirse ahora partiendo de las condiciones iniciales; por ejemplo, la característica que pasa por el punto $(-2,0)$ es $x=-2$ y la característica que pasa por $(2,0)$ es $x=2+t$. Entonces tenemos $u(-2,t)=0$ y $u(2+t,t)=1$, para $t>0$. De esta manera se ha obtenido la solución del problema en forma paramétrica.

















