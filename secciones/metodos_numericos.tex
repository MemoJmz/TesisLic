\chapter{Métodos Numéricos}\label{cap:numericos}

Cuando intentamos calcular soluciones numericamente nos enfrentamos a nuevos problemas. Esperamos que una discretización de diferencias finitas de la EDP sea inapropiada cerca de discontinuidades, donde la EDP no se mantiene. De hecho, al calcular soluciones discontinuas a leyes de conservación usando métodos estándares desarrollados bajo la hipótesis de soluciónes suaves, tipicmente obtenemos grandes oscilaciones numéricas incorrectas.

\textbf{Seguimiento del choque}. Desde que 

\section{Métodos Explícitos en Diferencias Finitas}
Para ecuaciones parciales hiperbólicas de dos variables dependientes que dependen de 2 varaibles independientes, el método de integrar a lo largo de curvas caractertísticas es usualmente la mejor manera para encontrar la solución.

Una de las ventajas de el método de las características es el hecho de que las discontinuidades se pueden propagar a través de las curvas características. Esto representa una dificultad para los métodos en diferencias finitas si las discontinuidades no caen dentro.

\textbf{Lo anterior no es del todo claro, hay que complementarlo}

Consideremos el problema de valor inicial
\[
\begin{cases}
u_tt = u_{xx}	\\
u(x,0) = f(x)	\\
u_t (x,0) = g(x)
\begin{cases}
\]
Nuestra intención ahora es construir un método numérico explícito.

Consideremos una malla que contiene a los puntos









