\chapter{Método de Características}\label{cap:características}

\section{}
El concepto de características fue introducido en $[?]$como vehículo para clasificar ecuaciones. En las secciones previas se mostró que ecuaciones hiperbólicas cuasilineales son sustancialmente simplificadas si las caracteríasticas son usadas. $[\textbf{AMES}]$ se refiere a ellas como el sistema de coordenadas natural. 

La razón básica que sustenta el uso de características es que, por una elección apropiada de coordenadas, el sistem orginal de ecuaciones hiperbólicas puede ser replanteado por un sistema cuyas coordenadas son las características.

Además, las simplificaciones son particularmente útiles cuando las aplicamos a ecuaciones de primer y segundo orden con dos variables independientes.

Nuestro principar interés es la ecuación cuasilineal de segundo orden
\[
a \thinspace u_{xx} + b \thinspace u_{xy} + c \thinspace u_{yy} = f
\]
donde $a,b,c$ y $f$ son funciones de $x,y,u, u_x, u_y$.


 